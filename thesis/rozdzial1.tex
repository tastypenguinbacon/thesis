\chapter{Wprowadzenie}
\label{cha:wprowadzenie}
    \section{Sieci neuronowe}
        Wraz z postępem w mocy obliczeniowej na popularności zyskały 
        sieci neuronowe. Ich zdolność do wykrywania wzorów i 
        prawidłowości znalazła zastosowanie między innymi w analizie
        obrazów \cite{neural-doodle}, rozpoznawaniu mowy oraz w sztucznej
        inteligencji prostych \cite{mari-o} oraz bardziej zaawansowanych
        \cite{alpha-go} botów. Przy odpowiednim procesie uczenia sieć
        neuronowa jest w stanie w znacznym stopniu przewyższyć eksperta 
        w danej dziedzinie \cite{alpha-go}.
        \\ \\
        Dodatkowym elementem przyśpieszającym rozwój sieci neuronowych
        jest powszechna dostępność zaawansowanych narzędzi takich jak 
        TensorFlow \cite{tensor-flow}, scikit-learn \cite{scikit-learn},
        Theano \cite{theano}. Tworzą one pewien poziom abstrakcji, 
        zarówno zmniejszając próg wejścia jak i pozwalając skupić się na 
        strukturze sieci neuronowej, zamiast na jej implementacji.
        \\ \\
        Obydwa te czynniki poskutkowały tym, że oprócz profesjonalnych
        zastosowań takich jak systemy rekomendacji portali
        społecznościowych powstało wiele amatorskich projektów o 
        otwartym kodzie źródłowym \cite{neural-doodle, mari-o, 
        coding-train} oraz zasobów wiedzy \cite{siraj, coding-train}.

    \section{AlphaGo}
        Pewnym przełomem w rozwoju sieci neuronowych był program AlphaGo,
        który wraz z postępem nauczania przewyższył najlepszego na 
        świecie gracza w Go. Dokonano tego wykorzystując głęboką sieć
        neuronową z warstwami konwolucyjnymi. Zastosowane algorytmy
        pozwalają na to aby AlphaGo ulegało samodoskonaleniu poprzez
        grę ze sobą.

    \section{,,Gra w życie''}
        ,,Gra w życie''\cite{stanford-gol} jest jednym z najbardziej 
        znanych automatów komórkowych posiadającym kopmletność Turinga.
        Jest to gra o stosunkowo prostych zasadach, które zastosowane
        dla wielu komórek skutkują w skomplikowanych, trudnych do 
        przewidzenia zachowaniach.
        \\ \\
        Gra występuje w wielu wariantach, między innymi z planszą
        skończoną, nieskończoną oraz zapętloną na krawędziach.

    \section{Cel pracy}
        Praca ma na celu skonstruowanie głębokiej sieci neuronowej, 
        która będzie w stanie dodawać komórki do planszy ,,Gry w życie''
        w taki sposób aby liczba komórek mieściła się w zadanym
        przedziale. Liczba komórek jakie mogą zostać dodane jest
        ograniczona od góry. 
        \\ \\
        Pod wieloma względami problemem podobnym do sieci neuronowej
        grającej zgodnie z tymi zasadami są sieci neurnowe nauczone gry
        w Go. Podobnie jak "Gra w życie", Go jest rozgrywane na planszy
        mającej postać siatki, a wynikiem analizy dokonanej przez sieć 
        neuronową jest położenie, gdzie ma zostać wykonany ruch ze
        strony sztucznej inteligencji \cite{alpha-go, toronto-go-1,
        toronto-go-2}.
        \\ \\
        Praca ta traktuje o zastosowaniu podobnych technik do nauczenia
        sieci neuronowej sterowania liczebnością komórek w ,,Grze w 
        życie''.
